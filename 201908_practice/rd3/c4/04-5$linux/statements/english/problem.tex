\begin{problem}{基因退化}{standard input}{standard output}{1 second}{512 megabytes}

人类的基因是一个由小写字母构成的字符串 $ S_1 $ 。由于近年来人们好吃懒作,他们的基因开始退化了!

慌张的科学家们发现他们不能阻止基因的退化,但他们发现了一些规律:基因中的一个已知的子序列 $ S_2 $ 会从基因当中消失,之后人类的剩下的基因就会散成好几段,而这几段中最长的一段会成为新的基因(其他较短的片段会被淘汰)。

幸运的是,我们现在知道了会消失的子序列是什么。现在想问,新基因可能具有的最长长度是多少。

\InputFile
第一行包含一个字符串, $ S_1 $ 代表退化之前的基因。

第二行包含一个字符串, $ S_2 $ 代表消失的子序列。

($ S_1 $、$ S_2 $的长度小于等于 100000)

\OutputFile
输出新基因可能具有的最长长度。

\Example

\begin{example}
\exmpfile{example.01}{example.01.a}%
\end{example}

\Note
样例解释:
bbaba有3种退化情况:

1.第一个b和第二个b消失剩下aba

2.第一个b和第三个b消失剩下ba和ba

3.第二个b和第三个b消失剩下b、a和a

因此新基因最长因该是aba长度为3

\end{problem}

